\documentclass[12pt]{article}

\pagenumbering{gobble}

\usepackage{amsfonts}
\usepackage{amsmath}
\usepackage{amssymb}
\usepackage{fancyhdr}
\usepackage[headheight=1in,margin=1in]{geometry}

\newcommand{\N}{\mathbb{N}}
\newcommand{\Z}{\mathbb{Z}}
\newcommand{\Q}{\mathbb{Q}}
\newcommand{\R}{\mathbb{R}}
\newcommand{\C}{\mathbb{C}}

\newcommand{\angleb}[1]{\left\langle#1\right\rangle}
\newcommand{\braceb}[1]{\left\{#1\right\}}
\newcommand{\bracketb}[1]{\left[#1\right]}
\newcommand{\parenb}[1]{\left(#1\right)}
\newcommand{\vertb}[1]{\left\vert#1\right\vert}

\newcommand{\done}{\ensuremath{\strut\hfill\blacksquare}}

\begin{document}

\pagestyle{fancy}
\fancyhead[L]{Algebra}
\fancyhead[R]{Symmetric Group}

\setlength{\parindent}{0in}
\setlength{\parskip}{0.1in}

\section*{The Symmetric Group}

The \textbf{symmetric group}, written \( S_n \), is the set of
all bijections on a set of order \( n \) under composition, where \( n \) is
typically finite.
Elements in \( S_n \) are typically written as permutations.
Take, for example, \( \sigma \in S_5 \) group, defined as
\[
	\sigma = \begin{pmatrix}
		1 & 2 & 3 & 4 & 5 \\
		3 & 2 & 5 & 1 & 4
	\end{pmatrix},
\]
which means \( \sigma(1) = 3 \), \( \sigma(2) = 2 \), \( \sigma(3) = 5\), etc.
We have that \( \vertb{S_n} = n! \).

\subsection*{Cycle Decomposition}

A \textbf{cycle} is an element of \( S_n \), written
\( (a_1, a_2, \dots, a_m) \), where \( a_i \in S_n \) and
\( a_i \mapsto a_{i + 1} \) with \( i \) mod \( m \).
All cycles are a product of disjoint cycles.
The \textbf{length} of a cycle is the number of non-identity cycles in its
disjoint cycle representation.
Disjoint cycles commute.

\textbf{Theorem:} Any element \( \sigma \in S_n \) has a unique factorization
into disjoint cycles up to rearrangement.

\textbf{Theorem:} The order of \( \sigma \in S_n \) is the least common
multiple of the length of the cycles in its disjoint cycle representation.

\section*{Matrix Groups}

Given a field \( F \), \( GL_n(F) \) is defined as
\[
	GL_n(F) = \braceb{A : A \in F^{n \times n}, \text{det}(A) \ne 0 }.
\]
\( GL_n(F) \) is a group under multiplication but not addition.

\section*{Extras}

\textbf{Definition:} A \textbf{group homomorphism} is a function between
two groups \( (G, *_G) \) and \( (H, *_H) \), defined \( f : G \to H \) where
\[
	f(a *_G b) = f(a) *_H f(b)
\]
for all \( a, b \in G \).

\end{document}
