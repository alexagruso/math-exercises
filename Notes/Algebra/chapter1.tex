\documentclass[12pt]{article}

\pagenumbering{gobble}

\usepackage{amsfonts}
\usepackage{amsmath}
\usepackage{amssymb}
\usepackage{fancyhdr}
\usepackage[headheight=1in,margin=1in]{geometry}

\newcommand{\N}{\mathbb{N}}
\newcommand{\Z}{\mathbb{Z}}
\newcommand{\Q}{\mathbb{Q}}
\newcommand{\R}{\mathbb{R}}
\newcommand{\C}{\mathbb{C}}

\newcommand{\angleb}[1]{\left\langle#1\right\rangle}
\newcommand{\braceb}[1]{\left\{#1\right\}}
\newcommand{\bracketb}[1]{\left[#1\right]}
\newcommand{\parenb}[1]{\left(#1\right)}
\newcommand{\vertb}[1]{\left\vert#1\right\vert}

\newcommand{\done}{\ensuremath{\strut\hfill\blacksquare}}

\begin{document}

\pagestyle{fancy}
\fancyhead[L]{Algebra}
\fancyhead[R]{Introduction}

\setlength{\parindent}{0in}
\setlength{\parskip}{0.1in}

\section*{Groups}

\textbf{Definition:} A group \( (G, *) \) is a set \( G \) and a binary
operation \( * : G \times G \to G \) that satisfies the following properties:

\begin{itemize}
	\item[a.)] For all \( a,b,c \in G \), we have
		\( a * (b * c) = (a * b) * c \).
		\hfill \textit{Associativity}

	\item[b.)] There exists \( e \in G \) where for all \( a \in G \) we have
		\( a * e = a \).
		\hfill \textit{Identity}

	\item[c.)] For all \( a \in G \), there exists \( a^{-1} \in G \) where
		\( a * a^{-1} = e \).
		\hfill \textit{Inverses}

	\item[d.)] For all \( a,b \in G \), we have \( a * b \in G \).
		\hfill \textit{Closure}
\end{itemize}

\textbf{Definition:} Let \( (G, *_G) \) and \( (H, *_H) \) be groups, then
\( (G \times H, *) \) is a group where \( * \) is defined as
\[
	(g, h) * (g', h') = (g *_G g', h *_H h')
\]
where \( (g', h'),(g', h') \in G \times H \).

\textbf{Proposition:} Let \( (G, *) \) be a group. Then the following are true:

\begin{itemize}
	\item[a.)] The identity \( e \in G \) is unique.

	\item[b.)] For all \( a \in G \), \( a^{-1} \) is unique.

	\item[c.)] For all \( a \in G \), \( \parenb{a^{-1}}^{-1} = a \).

	\item[d.)] For all \( a,b \in G \), we have \( (ab)^{-1} = b^{-1}a^{-1} \).
\end{itemize}

\textit{Proof of (a):} Suppose \( e,f \in G \) are both identities in \( G \),
then for \( a \in G \) we have
\[
	ae = a = af \implies a^{-1}ae = a^{-1}af \implies e = f,
\]
thus the identity is unique.
\done

\textit{Proof of (b):} Suppose \( b,b' \in G \) are both inverses of
\( a \in G \), then
\[
	ab = e = ab' \implies a^{-1}ab = a^{-1}ab' \implies b = b',
\]
thus inverses are unique.
\done

\textbf{Definition:} Suppose \( G \) is a group and let \( x \in G \), then
the \textbf{order} of \( x \) (written \( \vertb{x} \)), is the smallest
positive integer \( n \) where \( x^n = e \), where \( x^n \) denotes
\[
	\underbrace{x * x * \cdots * x}_{n \ \text{times}}.
\]
If no such \( n \) exists, we say \( x \) has infinite order and write
\( \vertb{x} = \infty \).

\section*{Dihedral Groups and Symmetries of Geometric Objects}

\textbf{Definition:} For \( n \in \Z_{\geq 3} \), we define
\( D_{2n} \) as the \textbf{dihedral group} of order \( 2n \), i.e. the
symmetries of an \( n \)-gon. These symmetries are described by certain
permutations of \( \braceb{1, 2, \dots, n} \). The group \( D_{2n} \) is
generated by \( r \), a rotation by \( 2\pi/n \), and \( s \), a reflection
across the vertical axis.

We can present \( D_{2n} \) using the elements \( r \) and \( s \):
\[
	D_{2n} = \angleb{r,s : rs = sr^{-1}},
\]
where \( r^{-1} = r^{n - 1} \). This is in terms of
``generators and relations''.

\end{document}
