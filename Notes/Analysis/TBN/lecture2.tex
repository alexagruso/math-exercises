\documentclass[12pt]{article}

\pagenumbering{gobble}

\usepackage{amsfonts}
\usepackage{amsmath}
\usepackage{amssymb}
\usepackage{fancyhdr}
\usepackage[headheight=1in,margin=1in]{geometry}
\usepackage[colorlinks=true,linkcolor=blue]{hyperref}
\usepackage{mathtools}

\newcommand{\N}{\mathbb{N}}
\newcommand{\Z}{\mathbb{Z}}
\newcommand{\Q}{\mathbb{Q}}
\newcommand{\R}{\mathbb{R}}
\newcommand{\C}{\mathbb{C}}

\newcommand{\e}{\varepsilon}

\newcommand{\angleb}[1]{\left\langle#1\right\rangle}
\newcommand{\braceb}[1]{\left\{#1\right\}}
\newcommand{\bracketb}[1]{\left[#1\right]}
\newcommand{\parenb}[1]{\left(#1\right)}
\newcommand{\vertb}[1]{\left\vert#1\right\vert}

\newcommand{\done}{\ensuremath{\strut\hfill\blacksquare}}

\begin{document}

\pagestyle{fancy}
\fancyhead[L]{Analysis}
\fancyhead[C]{Alex Agruso}
\fancyhead[R]{Notes}

\setlength{\parindent}{0in}
\setlength{\parskip}{0.1in}

\section*{Measure Theory}

\subsection*{Volume in \( \R^d \)}

A point \( x \in \R^d \) is a \( d \)-tuple of real numbers
\( (x_1, x_2, \dots, x_d) \) for \( x_i \in \R \).
The standard norm (Euclidian norm) \( \vertb{x} \) on \( \R^d \) is
defined as
\[
	\vertb{x}
	= \vertb{(x_1, x_2, \dots, x_d )}
	= \sqrt{x_1^2 + x_2^2 + \cdots + x_d^2}.
\]
Given two points \( x, y \in \R^d \), the distance between them is
\( \vertb{x - y} \).

The complement of \( E \subseteq \R^d \), written \( E^\complement \), is
defined as
\[
	E^\complement = \braceb{x \in \R^d : x \notin E}.
\]

If \( E, F \in \R^d \), then the complement of \( F \) in \( E \), written
\( E - F \) or \( E \backslash F \), is defined as
\[
	E - F = \braceb{e \in E : e \notin F}.
\]

The distance between two sets \( E, F \in \R^d \) is the function
\( d : \R^d \times \R^d \to \R \) defined as
\[
	d(E, F) = \inf_{(x, y) \in E \times F} \vertb{x - y}.
\]
This can be interpreted as the smallest distance between any two points in the
closures of \( E \) and \( F \).

\subsection*{Open, Closed, and Compact Sets}

The \textbf{Open Ball} of radius \( r \in \R \) centered at \( x \in \R^d \),
written \( B_r(x) \), is defined as
\[
	B_r(x) = \braceb{y \in \R^d : \vertb{y - x} < r}.
\]

A subset \( E \subseteq \R^d \) is \textbf{open} if for all \( x \in E \)
there exists real \( r > 0 \) where \( B_r(x) \subset E \).
\( E \) is closed if \( E^\complement \) is open.
Any union of open sets is open, the union need not be countable.
The intersection of finitely many open sets is open.
Conversely, arbitrary intersections of closed sets are closed, while only
finite unions of closed sets are guaranteed to be closed.

The \textbf{closure} of a set \( E \subseteq \R^d \), written
\( \overline{E} \), is defined as
\[
	\overline{E} = \bigcap_{\alpha \in \mathcal{A}} F_\alpha
\]
where \( \mathcal{A} \) is the family of all closed set \( F \in \R^d \) that
contain \( E \).

A set \( E \subseteq \R^d \) is \textbf{bounded} if there exists real
\( r > 0 \) and \( x \in \R^d \) where \( E \subset B_r(x) \).
A bounded set is \textbf{compact} if it is also closed.
Compact sets have the Heine-Borel covering property.

Assume \( E \subseteq \R^d \) is compact, and
\[
	E \subset \bigcup_{\alpha \in \mathcal{A}} O_\alpha
\]
where \( O_\alpha \) is open for all \( \alpha \).
Then there exists a finite subset \( \mathcal{B} \subseteq \mathcal{A} \)
where
\[
	E \subset \bigcup_{\beta \in \mathcal{B}} O_\beta
\]
In other words, every open covering of \( E \) contains a finite
subcovering.

A point \( x \in \R^d \) is a \textbf{limit point} of \( E \) if for all real
\( r > 0 \), \( B_r(x) \cap E \ne \varnothing \).

An \textbf{isolated point} in \( E \subseteq \R^d \) is a point \( x \in E \)
where for some real \( r > 0 \), we have that \( B_r(x) \cap E = \braceb{x} \).
If for some real \( r > 0 \), we have that \( B_r(x) \subset E \), then \( x \)
is an \textbf{interior point}.
The set of all interior points in \( E \), written \( E^\circ \) is called its
\textbf{interior}. Equivallently, the \textbf{closure} of \( E \) is the union
of \( E \) and its limit points.
The \textbf{boundary} of \( E \), written \( \partial E \), is defined as
\( \overline{E} \backslash E^\circ \).

A closed set \( E \subseteq \R^d \) is \textbf{perfect} if it has no isolated
points.

\subsubsection*{Rectangle and Cubes}

\textbf{Definition:} a rectangle \( R \subseteq R^d \) is given by the product
of \( d \) one-dimensional closed and bounded intervals, given by
\( R = [a_1, b_1] \times [a_2, b_2] \times \cdots \times [a_d. b_d] \) for
\( a_i, b_i \in \R \).
The volume of \( R \), written \( \vertb{R} \), is defined as
\[
	\vertb{R} = \prod_{i = 1}^d \vertb{a_i - b_i}
\]

\textbf{Definition:} a cube \( Q \subseteq \R^d \) is a rectangle whose
\( d \) intervals are of the same length. \( \vertb{Q} = \ell^d \) where
\( \ell \) is the length of the intervals.

A union of rectangles is \textbf{almost disjoint} if their interiors are
disjoint.

\textbf{Lemma 1.1:} If a rectangle is the almost disjoint union of finitely
many other rectangles, i.e.
\[
	R = \bigcup_{k = 1}^N R_k,
\]
then
\[
	\vertb{R} = \sum_{k = 1}^N \vertb{R_k}.
\]

\textbf{Lemma 1.2:} If \( R_1, R_2, \dots, R_n \subseteq \R^d \) are
rectangles and \( R \) is contained in their union, then we have
\[
	\vertb{R} \leq \sum_{k = 1}^N \vertb{R_k}.
\]

\textbf{Theorem:}
Every open subset \( O \) of \( \R \) can be written uniquely as a countable
union of disjoint open intervals.

\textit{Proof:}
For all \( x \in O \), let \( I_x \) denote the largest open interval such that
\( x \in I_x \subseteq O \).
Thus, if we define \( a_x = \inf \braceb{a : a < x, (a, x) \subseteq O } \) and
\( b_x = \sup \braceb{b : b > x, (x, b) \subseteq O }\), we must have that
\( a_x < x < b_x \).
Define \( I_x = (a_x, b_x) \).
We know that \( x \in I_x \) and \( I_x \subseteq O \), thus
\[
	O = \bigcup_{x \in O} I_x.
\]
If \( I_x \cap I_y \ne \varnothing \), then
\( x \in I_x \cup I_y \subseteq O \).
Since \( I_x \) is maximal, we have \( I_x \cup I_y \subseteq I_x \).
Similarly, \( I_x \cup I_y \subseteq I_y \), and thus \( I_x = I_y \).
Consequently, if \( \mathcal{I} = \braceb{I_x}_{x \in O } \), then any two
distinct intervals are disjoint.
To show \( \mathcal{I} \) is countable, we show that each interval contains a
rational number.
Since the intervals are disjoint, these numbers are distinct, and thus we can
form a bijection from the intervals to a countable set of rational numbers.
\done

If \( O \) is open and
\[
	O = \bigcup_{k = 1}^\infty I_j
\]
for disjoint intervals \( I_j \), then the measure of \( O \) is
\[
	m(O) = \sum_{j = 1}^\infty \vertb{I_j}.
\]
Thus, given a union of two open sets \( O_1 \cap O_2 \ne \varnothing \), we
have that
\[
	m(O_1 \cup O_2) = m(O_1) + m(O_2).
\]

\textbf{Theorem:}
For \( d \geq 1 \), we have that every open set \( O \subseteq \R^d \) can be
written as a countable union of almost disjoint closed cubes.

\textit{Proof:}
Form a grid in \( \R^d \) of closed cubes with side length 1 and integer
vertex coordinates.
For each cube \( Q \) in this grid, we keep it if \( Q \subset O \), remove it
if \( Q \subset O^\complement \), and consider it seperately if
\( Q \cap O \ne \varnothing \).
We then bisect these cubes to obtain side lengths \( 1 / 2 \) and repeat the
above process.
Gather all accepted cubes in \( \mathcal{Q} \).
Since this countable process acts on countable objects, we have that
\( \mathcal{Q} \) is countable.
If \( x \in O \), then there exists a cube of side length \( 2^{-N} \) that
contains \( x \) is is contained in \( O \).
This cube is either in \( \mathcal{Q} \) or is contained in a cube in
\( \mathcal{Q} \), thus
\[
	O = \bigcup_{Q \in \mathcal{Q}} Q.
\]
\done

We can define a measure \( m : \R^d \to \R \) where if
\[
	O = \bigcup_{j = 1}^\infty O_j
\]
then
\[
	m(O) = \sum_{j = 1}^\infty \vertb{O_j}
\]

\textbf{Definition:} The \textbf{Cantor} set is defined as
\( C = \bigcap_{n = 1}^\infty C_n \) where
\( C_0 = [0, 1], C_1 = [0, \frac{1}{3}] \cup [\frac{2}{3}, 1], \dots \)
Thus we have a sequence of compacts sets where
\[
	C_0 \supset C_1 \supset C_2 \supset \cdots
\]
\( C \) is closed and compact, and \( C \) is totally disconnected, which
means that for \( x, y \in C \), we have that there exists \( z \notin C \)
where \( x < z < y \).
\( C \) is perfect, which means it has no isolated points.
\( C \) has the cardinality of the continuum.

\subsection*{Exterior (Outer) Measure}

The exterior measure \( m_* \) assigns to any subset of \( \R^d \) a notion
of size.
\( m_* \) is generally not additive for unions of disjoint sets.
To achieve additivity, we must sacrifice the measurability of some sets.
\( m_* \) approximates \( E \subseteq \R^d \) ``from the outside''.
We define \( m_* \) as
\[
	m_*(E) = \inf \sum_{j = 1}^\infty \vertb{Q_j}
\]
where the infimum is taken over countable coverings of \( E \) with closed
cubes.

\end{document}
