\documentclass[12pt]{article}

\pagenumbering{gobble}

\usepackage{amsfonts}
\usepackage{amsmath}
\usepackage{amssymb}
\usepackage{fancyhdr}
\usepackage[headheight=1in,margin=1in]{geometry}
\usepackage[colorlinks=true,linkcolor=blue]{hyperref}
\usepackage{mathtools}

\newcommand{\N}{\mathbb{N}}
\newcommand{\Z}{\mathbb{Z}}
\newcommand{\Q}{\mathbb{Q}}
\newcommand{\R}{\mathbb{R}}
\newcommand{\C}{\mathbb{C}}

\newcommand{\e}{\varepsilon}

\newcommand{\angleb}[1]{\left\langle#1\right\rangle}
\newcommand{\braceb}[1]{\left\{#1\right\}}
\newcommand{\bracketb}[1]{\left[#1\right]}
\newcommand{\parenb}[1]{\left(#1\right)}
\newcommand{\vertb}[1]{\left\vert#1\right\vert}

\newcommand{\done}{\ensuremath{\strut\hfill\blacksquare}}

\begin{document}

\pagestyle{fancy}
\fancyhead[L]{Analysis}
\fancyhead[C]{Alex Agruso}
\fancyhead[R]{Notes}

\setlength{\parindent}{0in}
\setlength{\parskip}{0.1in}

\section*{The Exterior Measure}

\textbf{Definition:} Let \( E \subseteq \R^d \), then the
\textbf{exterior measure} \( m_* \) of \( E \) is defined as
\[
	m_*(E) = \inf \sum_{j = 1}^\infty \vertb{Q_j},
\]
where \( \braceb{Q_j} \) is a countable covering of \( E \) using closed cubes.

We have that \( m_*(E) \geq 0 \) for any \( E \subseteq \R^d \).
For any point \( x \in \R^d \), we have that \( m_*(\braceb{x}) = 0 \); since
any point is a closed cube, it covers itself and thus contains a countable
cover. Since \( \vertb{\braceb{x}} = 0 \), we have that
\( m_*(\braceb{x}) = 0 \).
We also have that \( m_*(\varnothing) = 0 \), as any covering covers
\( \varnothing \), and since some coverings have zero volume, we have that
\( m_*(\varnothing) \leq 0 \), and thus \( = 0 \).

Let \( Q \subset \R^d \) be a closed cube, then \( m_*(Q) = \vertb{Q} \).
We obviously have that \( m_*(Q) \leq \vertb{Q} \).
Now consider an arbitrary covering \( \mathcal{Q} \) where
\[
	Q \subseteq \bigcup_{j = 1}^\infty Q_j.
\]
Fix \( \varepsilon > 0 \), and for each \( Q_j \in \mathcal{Q} \), choose
an open cube \( S_j \) such that \( Q_j \subset S_j\) and
\( \vertb{S_j} \leq (1 + \varepsilon)\vertb{Q_j} \).
Since \( Q \) is compact and \( \braceb{S_j} \) is an open cover
of \( Q \), we have a finite subcover \( \mathcal{S} \subseteq \braceb{S_j} \).
We have that \( m_*() \)...

Let \( Q \subseteq \R^d \) be any open cube, then \( m_*(Q) = \vertb{Q} \).

Given a rectangle \( R \subset \R^d \), we have \( m_*(R) = \vertb{R} \).

\( m_*(\R^d) = \infty \).

\end{document}
