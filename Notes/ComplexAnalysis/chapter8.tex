\documentclass[12pt]{article}

\pagenumbering{gobble}

\usepackage{amsfonts}
\usepackage{amsmath}
\usepackage{amssymb}
\usepackage{fancyhdr}
\usepackage[headheight=1in,margin=1in]{geometry}
\usepackage[colorlinks=true,linkcolor=blue]{hyperref}
\usepackage{mathtools}
\usepackage{tcolorbox}

\tcbuselibrary{theorems}

\newcommand{\N}{\mathbb{N}}
\newcommand{\Z}{\mathbb{Z}}
\newcommand{\Q}{\mathbb{Q}}
\newcommand{\R}{\mathbb{R}}
\newcommand{\C}{\mathbb{C}}

\newcommand{\e}{\varepsilon}

\newcommand{\angleb}[1]{\left\langle#1\right\rangle}
\newcommand{\braceb}[1]{\left\{#1\right\}}
\newcommand{\bracketb}[1]{\left[#1\right]}
\newcommand{\parenb}[1]{\left(#1\right)}
\newcommand{\vertb}[1]{\left\vert#1\right\vert}

\newcommand{\proof}{\textit{Proof: }}
\newcommand{\altproof}{\textit{Alternate Proof: }}
\newcommand{\done}{\ensuremath{\strut\hfill\blacksquare}}

\newtcbtheorem[number within=section]{theorem}{Theorem}{
	boxsep=0.1in,
	colframe=blue!30!black,
	fonttitle=\bfseries
}{th}

\newtcbtheorem[number within=section]{lemma}{Lemma}{
	boxsep=0.1in,
	colframe=blue!30!black,
	fonttitle=\bfseries
}{th}

\newtcbtheorem[number within=section]{corollary}{Corollary}{
	boxsep=0.1in,
	colframe=blue!30!black,
	fonttitle=\bfseries
}{th}

\begin{document}

\pagestyle{fancy}
\fancyhead[L]{Complex Analysis}
\fancyhead[C]{}
\fancyhead[R]{Chapter 8 - Cauchy Integral Formula}

\setlength{\parindent}{0in}
\setlength{\parskip}{0.1in}

\setcounter{section}{8}

\begin{lemma}{}{}
	Let \( \Omega \subseteq \C \) be a domain such that \( \partial\Omega \) is
	a positively oriented simple closed contour.
	Then for \( z_0 \notin \partial\Omega \), we have that
	\[
		\int_{\partial\Omega} \frac{dz}{z - z_0}
		= \begin{cases}
			0 & \text{if} \ z_0 \notin \overline{\Omega}
			= \partial\Omega \\
			2\pi i & \text{if} \ z_0 \in \Omega
		\end{cases}
	\]
\end{lemma}

\proof
Suppose \( z_0 \notin \overline{\Omega} \), then for each
\( z \in \partial\Omega \) define \( r(z) \) such that
\( z_0 \notin D_{r(z)}(z) \).
Now define \( \Lambda \subseteq \C \) and \( f : \C \to \C \) as
\[
	\Lambda = \parenb{\bigcup_{z \in \partial\Omega} D_{r(z)}(z) } \cup \Omega
	\hspace{0.25in} \text{and} \hspace{0.25in}
	f(z) = \frac{1}{z - z_0}.
\]
We have that \( \Lambda \) is a domain since it is a union of open balls in
\( \C \), and thus since \( z_0 \notin \Lambda \), we have that \( f \) is
holomorphic in \( \Lambda \).
Since \( \partial\Omega \subseteq \Lambda \), we have that \( f \) is
holomorphic on \( \partial\Omega \), and thus by Cauchy's integral theorem, we
obtain
\[
	\int_{\partial\Omega} \frac{dz}{z - z_0}
	= \int_{\partial\Omega} f(z) \, dz
	= 0.
\]
Now suppose \( z_0 \in \Omega \), then choose real \( r > 0 \) such that
\( C = C_{r}(z_0) \subseteq \Omega \).
We can parameterize \( C \) by \( z = \omega(\theta) = z_0 + re^{i\theta} \)
where \( 0 \leq \theta \leq 2\pi \).
Consequently, we have that \( dz = ire^{i\theta} \, d\theta \), and thus
\[
	\int_C \frac{dz}{z - z_0}
	= \int_0^{2\pi} \frac{ire^{i\theta}}{z_0 + re^{i\theta} - z_0} \, d\theta
	= \int_0^{2\pi} \frac{ire^{i\theta}}{re^{i\theta}} \, d\theta
	= i \int_0^{2\pi} \, d\theta
	= 2\pi i.
\]
Since \( C \) is contained in the interior of \( \partial\Omega \), we have
that
\[
	\int_C \frac{dz}{z - z_0}
	= \int_{\partial\Omega} \frac{dz}{z - z_0}
	= 2\pi i.
\]
\done

\pagebreak
\begin{lemma}{}{}
	Let \( f \) be holomorphic in some domain \( \Omega \) and let
	\( C \subseteq \Omega \) be a positively oriented simple closed contour.
	Then, for all \( z_0 \) in the interior of \( C \) we have that
	\[
		\int_C \frac{f(z) - f(z_0)}{z - z_0} \, dz = 0
	\]
\end{lemma}

\proof
Define \( g \) as
\[
	g(z) = \frac{f(z) - f(z_0)}{z - z_0},
\]
then because \( f \) is holomorphic in \( \Omega \), we have that \( g \) is
holomorphic in \( \Omega \setminus \braceb{z_0} \).
Define \( g(z_0) = f'(z_0) \), we will show that \( g \) is continuous at
\( z_0 \).
Choose real \( r > 0 \) such that \( D_r(z_0) \subseteq \Omega \), and for
\( n \in \N \), define \( r_n = r / n \).
We can define a sequence \( \braceb{z_n} \) such that
\( z_n \in D_{r_n}(z_0) \) for all \( n \).
It is easy to show that \( \braceb{z_n} \to z_0 \).
Since \( f \) is holomorphic at \( z_0 \), we have that
\[
	f'(z_0)
	= \lim_{z \to z_0} \frac{f(z) - f(z_0)}{z - z_0},
\]
and thus
\[
	\lim_{n \to \infty} g(z_n)
	= \lim_{z_n \to z_0} \frac{f(z_n) - f(z_0)}{z_n - z_0}
	= f'(z_0),
\]
which gives us \( \lim_{z \to z_0} g(z) = g(z_0) = f'(z_0) \), showing that
\( g \) is continuous at \( z_0 \).
Since we also have that \( g \) is holomorphic on
\( \Omega \setminus \braceb{z_0} \), we can apply Cauchy's integral theorem on
\( g \) to obtain
\[
	\int_C g(z) \, dz
	= \int_C \frac{f(z) - f(z_0)}{z - z_0} \, dz
	= 0.
\]
\done

\altproof
Since \( f \) is holomorphic in \( \Omega \), it is continuous at \( z_0 \),
thus for all \( \e > 0 \), there exists \( \delta > 0 \) such that
\( z \in D_\delta(z_0) \implies f(z) \in D_\e(f(z_0)) \).
Fix \( \e \) and choose \( \delta \) such that
\( \overline{D}_\delta(z_0) \subseteq \Omega \).
Additionally, let \( \partial D \) be the boundary of
\( \overline{D}_\delta(z_0) \).
By the ML inequality, we have that
\[
	\vertb{I}
	= \vertb{\int_{\partial D} \frac{f(z) - f(z_0)}{z - z_0} \, dz}
	\leq \int_{\partial D} \vertb{\frac{f(z) - f(z_0)}{z - z_0}} \, dz
	% = \int_{\partial D} \frac{|f(z) - f(z_0)|}{|z - z_0|} \, dz
	< \frac{\e}{\delta}L(\partial D)
	= \frac{\e}{\delta}2\pi\delta
	= 2\pi\e,
\]
and since \( 2\pi\e \) can be arbitrarily small, we have that \( I = 0 \).
Additionally, since \( z_0 \) is on the interior of \( C \), we have that
\( \partial D \) is contained in the interior of \( C \) for small enough
\( \delta \), and thus
\[
	\int_{C} \frac{f(z) - f(z_0)}{z - z_0} \, dz
	= \int_{\partial D} \frac{f(z) - f(z_0)}{z - z_0} \, dz
	= 0.
\]
\done

\begin{theorem}{Cauchy's Integral Formula}{}
	Let \( f \) be holomorphic in some domain \( \Omega \) and
	\( C \subseteq \Omega \) be a positively oriented simple closed contour.
	Then, for all \( z_0 \) in the interior of \( C \), we have that
	\[
		f(z_0)
		= \frac{1}{2\pi i} \int_C \frac{f(z)}{z - z_0} \, dz
	\]
\end{theorem}

\proof
We have that
\[
	\frac{1}{2\pi i} \int_C \frac{f(z_0)}{z - z_0} \, dz
	= \frac{f(z_0)}{2\pi i} \int_C \frac{dz}{z - z_0}
	= \frac{f(z_0)}{2\pi i} 2\pi i
	= f(z_0)
\]
and
\[
	\frac{1}{2\pi i} \int_C \frac{f(z) - f(z_0)}{z - z_0} \, dz = 0,
\]
thus
\[
	\frac{1}{2\pi i} \int_C \frac{f(z)}{z - z_0} \, dz
	= \frac{1}{2\pi i} \int_C \frac{f(z) - f(z_0) + f(z_0)}{z - z_0} \, dz
\]
\[
	= \frac{1}{2\pi i} \int_C \frac{f(z) - f(z_0)}{z - z_0} \, dz
	+ \frac{1}{2\pi i} \int_C \frac{f(z_0)}{z - z_0} \, dz
	= 0 + f(z_0) = f(z_0).
\]
\done

\begin{corollary}{Gauss' Mean Value Theorem}{}
	Let \( f \) be holomorphic in some domain \( \Omega \) and for
	\( z_0 \in \Omega \) and real \( r > 0 \), let
	\( C = C_r(z_0) \subseteq \Omega \) be a positively oriented circle.
	Then we have that
	\[
		f(z_0)
		= \frac{1}{2\pi} \int_0^{2\pi} f(z_0 + re^{i\theta}) \, d\theta
	\]
\end{corollary}

\proof
We have that
\[
	f(z_0)
	= \frac{1}{2\pi i} \int_C \frac{f(z)}{z - z_0} \, dz,
\]
and parameterizing \( C \) with \( z = \omega(\theta) = z_0 + re^{i\theta} \)
for
\( 0 \leq \theta \leq 2\pi \), we obtain \( dz = ire^{i\theta} \, d\theta \),
thus
\[
	f(z_0)
	= \frac{1}{2\pi i} \int_0^{2\pi} \frac{f(z_0 + re^{i\theta})}{re^{i\theta}}
	ire^{i\theta} \, d\theta
	= \frac{1}{2\pi} \int_0^{2\pi} f(z_0 + re^{i\theta}) \, d\theta.
\]
\done

\begin{lemma}{}{}
	Let \( C \) be a contour and \( g \) continuous on \( C \).
	Also define \( f \) as
	\[
		f(z_0)
		= \frac{1}{2\pi i} \int_C \frac{g(z)}{z - z_0} \, dz.
	\]
	If \( R \) is a region where \( C \cap R = \varnothing \), then we have
	that \( f \) is holomorphic in \( R \) and that for all \( z_0 \in R \),
	\[
		f'(z_0) = \frac{1}{2\pi i} \int_C \frac{g(z)}{(z - z_0)^2} \, dz.
	\]
\end{lemma}

\proof
Fix real \( r > 0 \) such that \( D_r(z_0) \subseteq R \), and let
\( \braceb{z_n} \subseteq D_r(z_0) \) be a sequence that converges to \( z_0 \).
We have to show that \( \lim_{n \to \infty} A_n = 0 \), where
\[
	A_n = \frac{f(z_n) - f(z_0)}{z_n - z_0}
	- \frac{1}{2\pi i} \int_C \frac{g(z)}{(z - z_0)^2} \, dz
\]
Given our definition of \( f \), we have that
\begin{align*}
	\frac{f(z_n) - f(z_0)}{z_n - z_0}
	& = \frac{1}{2\pi i} \int_C \frac{g(z)}{z_n - z_0}
	\bracketb{\frac{1}{z - z_n} - \frac{1}{z - z_0}}
	\, dz \\
	& = \frac{1}{2\pi i} \int_C \frac{g(z)}{z_n - z_0} \cdot
	\frac{z_n - z_0}{(z - z_n)(z - z_0)}
	\, dz \\
	& = \frac{1}{2\pi i} \int_C \frac{g(z)}{(z - z_n)(z - z_0)}
	\, dz,
\end{align*}
and thus
\begin{align*}
	A_n & = \frac{1}{2\pi i} \int_C \frac{g(z)}{(z - z_n)(z - z_0)} \, dz
	- \frac{1}{2\pi i} \int_C \frac{g(z)}{(z - z_0)^2} \, dz \\
		& = \frac{1}{2\pi i} \int_C g(z)
		\bracketb{\frac{1}{(z - z_0)(z - z_n)} - \frac{1}{(z - z_0)^2}}
		\, dz \\
		& = \frac{1}{2\pi i} \int_C g(z)
		\frac{z_n - z_0}{(z - z_0)^2(z - z_n)} \, dz \\
		& = \frac{z_n - z_0}{2\pi i} \int_C
		\frac{g(z)}{(z - z_0)^2(z - z_n)} \, dz.
\end{align*}
Define \( M \) and \( D \) as 
\[
	M = \max_{z \in C} \vertb{g(z)}
	\hspace{0.2in} \text{and} \hspace{0.2in}
	D = \min_{w \in C} \vertb{z - z_0}.
\]
Choose \( n \in \N \) such that \( \vertb{z_0 - z_n} < D / 2 \), then we have
that
\[
	\vertb{A_n}
	= \vertb{
		\frac{z_n - z_0}{2\pi i} \int_C \frac{g(z)}{(z - z_0)^2(z - z_n)} \, dz
	}
	\leq \vertb{\frac{z_n - z_0}{2\pi i}} \cdot \frac{ML(C)}{D^2 \cdot D / 2}
	= \vertb{z_n - z_0} \frac{ML}{\pi D^3},
\]
but we know that \( \lim_{n \to \infty} \vertb{z_n - z_0} = 0 \), thus
\( \lim_{n \to \infty} \vertb{A_n} = 0 \).
\done

\end{document}
