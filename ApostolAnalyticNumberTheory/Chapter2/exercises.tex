\documentclass[12pt]{article}

\usepackage{amsfonts}
\usepackage{amsmath}
\usepackage{amssymb}
\usepackage{cite}
\usepackage{enumerate}
\usepackage{fancyhdr}
\usepackage[headheight=1in,margin=1.25in]{geometry}
\usepackage[colorlinks=true,linkcolor=blue]{hyperref}
\usepackage{mathtools}
\usepackage{setspace}

\usepackage{tikz}

\newcommand{\N}{\ensuremath{\mathbb{N}}}
\newcommand{\Z}{\ensuremath{\mathbb{Z}}}
\newcommand{\Q}{\ensuremath{\mathbb{Q}}}
\newcommand{\R}{\ensuremath{\mathbb{R}}}
\newcommand{\C}{\ensuremath{\mathbb{C}}}

\newcommand{\e}{\ensuremath{\varepsilon}}
\renewcommand{\d}{\ensuremath{\delta}}

\newcommand{\angleb}[1]{\left\langle#1\right\rangle}
\newcommand{\braceb}[1]{\left\{#1\right\}}
\newcommand{\bracketb}[1]{\left[#1\right]}
\newcommand{\parenb}[1]{\left(#1\right)}
\newcommand{\vertb}[1]{\left\vert#1\right\vert}
\newcommand{\dvertb}[1]{\left\Vert#1\right\Vert}

\DeclarePairedDelimiter\floor{\lfloor}{\rfloor}
\DeclarePairedDelimiter\ceil{\lceil}{\rceil}

\newcommand{\comp}{\complement}
\newcommand{\sdiff}{\setminus}

\newcommand{\proof}{\textit{Proof: }}
\newcommand{\partialdone}{\ensuremath{\strut\hfill\blacktriangle}}
\newcommand{\done}{\ensuremath{\strut\hfill\blacksquare}}

\newcommand{\mc}[1]{\ensuremath{\mathcal{#1}}}

\renewcommand{\t}[1]{\text{ #1 }}
\newcommand{\impl}{\ensuremath{\implies}}
\newcommand{\sectionskip}{\vspace{0.15in}}
\newcommand{\tri}{\triangle}

\begin{document}

\pagestyle{fancy}
\fancyhead[L]{Ahlfors}
\fancyhead[C]{Complex Analysis}
\fancyhead[R]{Chapter 1}

\setlength{\parindent}{0in}
\setlength{\parskip}{0.1in}
\setstretch{1}

\textbf{1.)} For each equation below, find all solutions for \( n \in \N \).
\begin{itemize}
	\item [(a)] \( \phi(n) = n / 2 \)
	\item [(b)] \( \phi(n) = \phi(2n) \)
	\item [(c)] \( \phi(n) = 12 \)
\end{itemize}

\textit{Solution:}
We first show that (a) is true if and only if \( n \) is a power of two.
Fix \( n = 2^m \) where \( m \in \N \), then
\[
	\phi(n) = \phi(2^m) = 2^m - 2^{m - 1} = 2^{m - 1} = n / 2.
\]
Next, suppose \( n \) is not a power of two.
If \( n \) is odd, then \( n / 2 \) is not an integer, and thus cannot be the
result of \( \phi(n) \).
This forces \( n \) to be even.
Let \( \prod_{i \in \N} p_i^{a_i} \) be the prime factorization of \( n \).
Since \( \phi \) is multiplicative, and since powers of distinct primes are
relatively prime, we have that
\[
	\phi(n)
	= \phi\parenb{\prod_{i \in \N} p_i^{a_i}}
	= \prod_{i \in \N} \phi(p_i^{a_i}).
\]
Additionally, because \( n \) is even, we know that \( a_1 \geq 1 \), and since
it is not a power of two, we know there exists \( i \) where \( p_i > 2 \) and
\( a_i \geq 1 \).
We also know that \( \phi(p_i^{a_i}) < p_i^{a_i} \) for all \( i \), thus we
have
\[
	\phi(n)
	= \prod_{i \in \N} \phi(p_i^{a_i})
	= \phi(2^{a_1}) \prod_{i \geq 2} \phi(p_i^{a_i})
	= 2^{a_1 - 1} \prod_{i \geq 2} \phi(p_i^{a_i})
	< \frac{2^{a_1} \prod_{i \geq 2} p_i^{a_i}}{2}
\]
\[
	= \frac{\prod_{i \in \N} p_i^{a_i}}{2}
	= \frac{n}{2}.
\]
The inequality is strict, thus showing that \( \phi(n) \ne n/2 \).
This proves the case of (a).
\partialdone

Next, we show that (b) holds only for \( n \) that are relatively prime to 2.
Let \( n \in \N \) and assign \( d = (2,n) \).
If \( d = 1 \), the multiplicity of \( \phi \) gives us
\[
	\phi(2n)
	= \phi(2)\phi(n)
	= \phi(n).
\]
Otherwise, assume \( d \ne 1 \).
Since \( d \) must divide 2, we know that \( d = 2 \), thus
\[
	\phi(2n)
	= \phi(2)\phi(n)\frac{d}{\phi(d)}
	= \phi(n)\frac{2}{\phi(2)}
	= 2\phi(n)
	\ne \phi(n).
\]
This holds because \( \phi(n) > 0 \) for all \( n \in \N \), thus proving (b).
\done

\end{document}
