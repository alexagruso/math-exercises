\documentclass[12pt]{article}

\usepackage{amsmath}
\usepackage{amsfonts}
\usepackage{amssymb}
\usepackage{fancyhdr}
\usepackage[headheight=0.25in,margin=1in]{geometry}

\newcommand{\parens}[1]{\left( #1 \right)}

\newcommand{\N}{\mathbb{N}}
\newcommand{\Z}{\mathbb{Z}}
\newcommand{\Q}{\mathbb{Q}}
\newcommand{\R}{\mathbb{R}}
\newcommand{\C}{\mathbb{C}}

\newcommand{\solution}{\textbf{Solution:}}
\newcommand{\proof}{\textbf{Proof:}}
\newcommand{\done}{\ensuremath{
    \strut\hfill\blacksquare
}}

\linespread{1.25}

\begin{document}
    \pagestyle{fancy}
    \fancyhead[L]{Chapter 2}
    \fancyhead[R]{Exercises}

    \begin{itemize}
<<<<<<< HEAD:analytic_number_theory/chapter_2/exercises.tex
        \item[1.)] Find all integers \( n \in \N \) that satisfy the following:
=======
        \item[1.)] For each of the following statements, find all integers
        \( n \) that satisfy them:
>>>>>>> 8f48256 (update):apostol_analytic_number_theory/chapter_2/exercises.tex
        \begin{itemize}
            \item [a.)] \( \phi(n) = n / 2 \)
            \item [b.)] \( \phi(n) = \phi(2n) \)
            \item [c.)] \( \phi(n) = 12 \)
        \end{itemize}

        \solution

        \begin{itemize}
            \item [a.)] \( n = 2^m \) where \( m \in \N \)

            \item [b.)] \( (2, n) = 1 \)
        \end{itemize}

        \proof

        \begin{itemize}
            \item [a.)] Let \( n = 2^m \) where \( m \in \N \), thus
            \[
<<<<<<< HEAD:analytic_number_theory/chapter_2/exercises.tex
                \phi(2^m)
                = 2^m - 2^{m - 1}
                = 2^{m - 1}
                = n / 2
                .
=======
                \phi(n) = \phi(2^m) = 2^m - 2^{m - 1} = 2^{m - 1} = n / 2
>>>>>>> 8f48256 (update):apostol_analytic_number_theory/chapter_2/exercises.tex
            \]
            Next, suppose \( n \) is not a power of two.
            If \( n \) is odd, then \( \phi(n) \ne n / 2 \) as the codomain
            of \( \phi \) is \( \N \), thus \( n \) must be even.
            Take \( \prod p_i^{a_i} \) to be the prime factorization of
            \( n \) extended over all primes.
            Since \( \phi \) is multiplicative, we find that
            \[
                \phi(n)
                = \phi \parens{\prod p_i^{a_i}}
                = \prod \phi(p_i^{a_i})
                .
            \]
            Since \( n \) is even, we know that \( a_1 \geq 1 \), and since it
            is not a power of two, we know there exists \( i \) where
            \( p_i > 2 \) and \( a_i \geq 1 \).
            Let \( b_i = \phi(p_i^{a_i}) \). If \( a_i = 0 \), thus
            \( b_i = 1 \), otherwise
            \( b_i = p_i^{a_i} - p_i^{a_i - 1} < p_i^{a_i} \), thus
            \( b_1b_2b_3 < n \).
            Using this, we find that
            \[
                \phi(n) = b_1 b_2 b_3 \cdots
                = 2^{a_i - 1} b_2 b_3 \cdots
                = \frac{2^{a_1} b_2 b_3 \cdots}{2}
                < \frac{2^{a_1} 3^{a_2} 5^{a_3} \cdots}{2}
                = \frac{n}{2}
                ,
            \]
            thus \( \phi(n) \ne n / 2 \).
            \done

            \item [b.)] Let \( n \in \N \) where \( (2, n) = 1 \).
            Since \( \phi(2) = 1 \), we have that
            \[
                \phi(2n)
                = \phi(2)\phi(n)
                = \phi(n)
                .
            \]
            Otherwise, if \( (2, n) = d \ne 1 \), then \( d = 2 \), thus
            \[
                \phi(2n)
                = \phi(2)\phi(n)\frac{d}{\phi(d)}
                = \phi(n)\frac{2}{\phi(2)}
                = 2\phi(n)
                \ne \phi(n)
                ,
            \]
            which is true because \( \phi(n) > 0 \) for all \( n \in \N \).
            \done

        \end{itemize}

        \item [2.)] Prove or disprove the following statements:
        \begin{itemize}
            \item [a.)] If \( (m, n) = 1 \), then
            \( (\phi(m), \phi(n)) = 1 \)

            \item [b.)] If \( n \) is composite, then \( (n, \phi(n)) > 1 \)
            
            \item [c.)] If \( m \) and \( n \) have the same prime divisors,
            then \( n\phi(m) = m\phi(n) \)
        \end{itemize}

        \item [3.)] Prove that
        \[
            \frac{n}{\phi(n)}
            = \sum_{d \mid n} \frac{\mu^2(d)}{\phi(d)}
        \]
    \end{itemize}
\end{document}