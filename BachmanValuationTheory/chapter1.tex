\documentclass[12pt]{article}

\usepackage{amsfonts}
\usepackage{amsmath}
\usepackage{amssymb}
\usepackage{fancyhdr}
\usepackage[headheight=1in,margin=1in]{geometry}

\newcommand{\N}{\mathbb{N}}
\newcommand{\Z}{\mathbb{Z}}
\newcommand{\Q}{\mathbb{Q}}
\newcommand{\R}{\mathbb{R}}
\newcommand{\C}{\mathbb{C}}

\newcommand{\braceb}[1]{\left\{#1\right\}}
\newcommand{\bracketb}[1]{\left[#1\right]}
\newcommand{\parenb}[1]{\left(#1\right)}
\newcommand{\vertb}[1]{\left\vert#1\right\vert}

\newcommand{\done}{\ensuremath{\strut\hfill\blacksquare}}

\begin{document}

\pagestyle{fancy}
\fancyhead[L]{p-Adic Numbers and Valuation Theory}
\fancyhead[R]{Bachman}

\setlength{\parindent}{0in}
\setlength{\parskip}{0.25in}

\section{Valuations of Rank One}

\subsection{p-Adic Valuations of \( \Q \)}

The absolute value is an example of a rank one valuation.
For example, the absolute value of a rational number satisfies the following
properties:

\begin{itemize}
	\item[1.)] Non-degeneracy:
	\[
		x \in \Q \implies \vertb{x} \geq 0
		\ \text{and}\ 
		\vertb{x} = 0 \iff x = 0
	\]

	\item[2.)] Multiplicative:
	\[
		x,y \in \Q \implies \vertb{x} \vertb{y} = \vertb{xy}
	\]

	\item[3.)] Triangle Inequality:
	\[
		x,y \in \Q \implies \vertb{x + y} \leq \vertb{x} + \vertb{y}
	\]
\end{itemize}


We will now motivate the construction of a stronger valuation on \( \Q \).

\textbf{Theorem:} Consider a rational number \( x \).
Fixing some prime number \( p \), we can represent \( x \) as follows:
\[
	x = p^\alpha \frac{a}{b},
\]
where \( a,b,\alpha \in \Z \), \( p \nmid a \), and \( p \nmid b \).

\textit{Proof:} Fix a prime number \( p \) and \( x \in \Q \) where
\( x \ne 0 \) and \( x = c/d \) for \( c,d \in \Z \).
Let \( \beta_1 \) and \( \beta_2 \) be the highest powers of \( p \) that
divide \( c \) and \( d \) respectively, then
\[
	x = \frac{c}{d} = \frac{p^{\beta_1}a}{p^{\beta_2}b}
	= p^{\beta_1 - \beta_2} \frac{a}{b}.
\]
Clearly \( \beta_1 - \beta_2 \in \Z \), and \( p \nmid a \) and
\( p \nmid b \), so we have obtained our desired form for \( x \).
\done


\end{document}
