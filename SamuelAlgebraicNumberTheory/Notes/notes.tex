\documentclass[12pt]{article}

\usepackage{amsfonts}
\usepackage{amsmath}
\usepackage{amssymb}
\usepackage{cite}
\usepackage{enumerate}
\usepackage{fancyhdr}
\usepackage[headheight=1in,margin=1.25in]{geometry}
\usepackage[colorlinks=true,linkcolor=blue]{hyperref}
\usepackage{mathtools}
\usepackage{scalerel}
\usepackage{setspace}

\usepackage{tikz}

\newcommand{\N}{\ensuremath{\mathbb{N}}}
\newcommand{\Z}{\ensuremath{\mathbb{Z}}}
\newcommand{\Q}{\ensuremath{\mathbb{Q}}}
\newcommand{\R}{\ensuremath{\mathbb{R}}}
\newcommand{\C}{\ensuremath{\mathbb{C}}}

\newcommand{\e}{\ensuremath{\varepsilon}}
\renewcommand{\d}{\ensuremath{\delta}}

\newcommand{\angleb}[1]{\left\langle#1\right\rangle}
\newcommand{\braceb}[1]{\left\{#1\right\}}
\newcommand{\bracketb}[1]{\left[#1\right]}
\newcommand{\parenb}[1]{\left(#1\right)}
\newcommand{\vertb}[1]{\left\vert#1\right\vert}
\newcommand{\dvertb}[1]{\left\Vert#1\right\Vert}

\DeclarePairedDelimiter\floor{\lfloor}{\rfloor}
\DeclarePairedDelimiter\ceil{\lceil}{\rceil}

\renewcommand{\Re}{\text{Re}}
\renewcommand{\Im}{\text{Im}}

\newcommand{\comp}{\complement}
\newcommand{\sdiff}{\setminus}

\newcommand{\solution}{\textit{Solution: }}
\newcommand{\proof}{\textit{Proof: }}
\newcommand{\partialdone}{\ensuremath{\strut\hfill\blacktriangle}}
\newcommand{\done}{\ensuremath{\strut\hfill\blacksquare}}

\newcommand{\mc}[1]{\ensuremath{\mathcal{#1}}}

\renewcommand{\t}[1]{\text{ #1 }}
\newcommand{\impl}{\ensuremath{\implies}}
\newcommand{\sectionskip}{\vspace{0.15in}}
\newcommand{\tri}{\triangle}
\newcommand{\ovl}[1]{\ensuremath{\overline{#1}}}

\begin{document}

\pagestyle{fancy}
\fancyhead[L]{Samuel}
\fancyhead[C]{Algebraic Number Theory}
\fancyhead[R]{Chapter 1}

\setlength{\parindent}{0in}
\setlength{\parskip}{0.1in}
\setstretch{1}

\section*{Preliminaries}

We define the natural numbers \( \N \) to be the strictly positive integers.
That is, \( \N = \braceb{n \in \Z : n > 0} \).
We additionally define \( \N_0 = \N \cup \braceb{0} \).

Unless otherwise stated, all rings are commutative and contain a 1.

Given a ring \( R \) and two ideals \( I \) and \( J \) of \( R \), define
their sum and product as follows:
\[
	I + J = \braceb{a + b : i \in a \t{and} b \in J},
\]
and
\[
	IJ = \braceb{
		\sum_{k = 1}^n a_kb_k : a_k \in I, b_k \in J, \t{and} n \in \N
	}.
\]
We can generalize the product of ideals to finite collections of ideals.
Let \( I_1, I_2, \dots, I_m \) be ideals of a ring \( R \), then we define
their product as follows:
\[
	I_1I_2 \cdots I_m = \braceb{
		\sum_{k = 1}^n \parenb{\prod_{l = 1}^m a_{k,l}}
		: a_{k,l} \in I_k \t{and} n \in \N
	}
\]
As a special case, the power ideal \( I^m \) is defined for \( m \in \N \) as
follows:
\[
	I^m = \braceb{
		\sum_{k = 1}^n \parenb{\prod_{l = 1}^m a_{k,l}}
		: a_{k,l} \in I \t{and} n \in \N
	}
\]

Given a ring \( R \) and an arbitrary indexing set \( I \), we define
\( R^{(I)} \) as the set of collections \( \braceb{a_i}_{i \in I} \) of
elements of \( R \) such that \( a_i = 0 \) for almost all \( i \in I \).
We can endow this set with an \( R \)-module structure by defining
componentwise addition and scalar multiplication.

We define \( \d_{i,j} \) as follows:
\[
	\d_{i,j} = \begin{cases}
		1 & \t{if} i = j  \\
		0 & \t{otherwise}
	\end{cases}
\]
Additionally, for the \( R \)-module \( R^{(I)} \), we define the elements
\( e_i = \braceb{d_{i,j}}_{j \in I} \) and denote
\( \braceb{e_i}_{i \in I} \) as the canonical basis for \( R^{(I)} \).

Given an \( R \)-module \( M \), it is understand that \( 0 \) serves as the
identity element of the abelian group \( M \).

\section*{1.1 Divisibility in Principal Ideal Rings}

\textbf{Proposition 1.1:}
\( \Z \) is a principal ideal domain.

\proof
Let \( I \) be a nonzero ideal in \( \Z \), then we can choose a nonzero
element \( b \in I \).
If \( b < 0 \), then \( b^2 > 0 \) and \( I \) has a positive element, thus
without loss of generality we can assume \( b > 0 \).
By the well ordering principle, we can also assume that \( b \) is minimal
among the positive integers in \( I \).
Now, fix an arbitrary element \( x \in I \).
Since \( \Z \) is a Euclidean domain, we have unique integers \( q \) and
\( r \) such that \( x = qb + r \), and where \( 0 \leq r < b \).
Consequently, we have that \( r = x - qb \in I \) since \( I \) is closed under
addition.
If \( r > 0 \), then we have found a positive integer in \( I \) that is
less than \( b \), a contradiction to our assumption, hence we must have
\( r = 0 \).
This shows that all elements in \( I \) are multiples of \( b \), and thus
\( I = (b) \) and is a principal ideal.
\done

\textbf{Proposition 1.2:}
If \( K \) is a field, then \( K[X] \) is a principal ideal domain.

\proof
Fix a nonzero ideal \( I \) in \( K[X] \).
Let \( d \) be the minimal nonzero degree of any element in \( I \), and let
\( b \) be an element in \( I \) with degree \( d \).
Letting \( x \) be any element in \( I \), we can use the fact that \( K[X] \)
is a Euclidean domain to find unique elements \( q, r \in I \) where
\( x = qb + r \), and where \( 0 \leq \deg(r) < \deg(b) \).
By closure, we have that \( r = x - qb \in I \), which combined with our
assumption on \( \deg(b) \) forces \( \deg(r) = 0 \).
Thus, any element \( x \in I \) is a multiple of \( b \), meaning that
\( I = (b) \) and is thus principal.
\done

\section*{1.2 Diophantine Equations}

\textbf{Theorem 1.3:}
We have that \( x, y, z \in \N \) satisfy the equation \( x^2 + y^2 = z^2 \)
if and only if there exists an integer \( d \) and relatively prime integers
\( u \) and \( v \) such that, after possible rearrangement of \( x \) and
\( y \):
\[
	x = d(u^2 - v^2), \quad
	y = 2duv, \quad\text{and}\quad
	z = d(u^2 + v^2).
\]

\proof
The backwards direction is easy to see, as
\[
	x^2 + y^2
	= [d(u^2 - v^2)]^2 + (2duv)^2
	= d^2(u^4 - 2u^2v^2 + v^4) + d^2(4u^2v^2)
\]
\[
	= d^2(u^4 + 2u^2v^2 + v^4)
	= d^2(u^2 + v^2)^2
	= z^2.
\]
Conversely, let \( x, y, z \in \N \) satisfy the equation.
We can, without loss of generality, assume that they are pairwise relatively
prime, since otherwise we can divide out by their gcd.
\partialdone

\textbf{Theorem 1.4:}
There exist no integers \( x, y, z \in \N \) that satisfy the equation
\[
	x^4 + y^4 = z^2.
\]

\proof
\partialdone

\textbf{Corollary 1.5:}
There exist no integers \( x, y, z \in \N \) that satisfy the equation
\[
	x^4 + y^4 = z^4.
\]

\proof
\partialdone

\section*{1.3 Lemmas on Ideals and Euler's \( \phi \)-function}

\textbf{Proposition 1.6:}
Fix natural numbers \( q \) and \( n \), and denote \( \tilde{q} \) as the
residue class \( q + n\Z \).
We have that the following are equivalent:
\begin{enumerate}[(a)]
	\item \( \gcd(q,n) = 1 \)

	\item \( \tilde{q} \) is a unit in the ring \( \Z/n\Z \)

	\item \( \tilde{q} \) generates the additive group \( \Z/n\Z \)
\end{enumerate}
As a corollary, we have that \( \phi(n) \) is equal to the number of units
in the ring \( \Z/n\Z \), as well as the number of generators of the additive
group \( \Z/n\Z \).

\proof
We will prove that (a) \impl (b) \impl (c) \impl (a).
We also choose \( q \) to be the unique representative of \( \tilde{q} \)
where \( 0 \leq q \leq n - 1 \).

First, assume (a).
By Bezout's lemma, there exist integers \( x \) and \( y \) such that
\( qx + ny = 1 \).
We thus have that \( qx \equiv 1 \pmod{n} \), which is equivalent to
\( \tilde{q} \cdot \tilde{x} = \tilde{1} \).
This proves (a) \impl (b), and so we now assume that \( \tilde{q} \) is a unit
in \( \Z/n\Z \).
Choose \( \tilde{x} \in \Z/n\Z \) such that
\( \tilde{q} \cdot \tilde{x} = \tilde{1} \), then for arbitrary
\( \tilde{a} \in \Z/n\Z \) we have
\( \tilde{a} = \tilde{a} \cdot \tilde{x} \cdot \tilde{q} \).
This is equivalent to \( \tilde{a} = (ax)\tilde{q} \), which shows that
\( \tilde{q} \) generates all elements in \( \Z/n\Z \), and (b) \impl (c).
Finally, assume \( \tilde{q} \) to be a generator for \( \Z/n\Z \), then
there exists an integer \( x \) where \( x\tilde{q} = \tilde{1} \), but this
means that \( xq \equiv 1 \pmod{n} \).
We can choose an integer \( y \) such that \( xq - 1 = yn \), which implies
that \( xq - yn = 1 \).
By Bezout's lemma, this implies that \( \gcd(q,n) = 1 \), thus showing
(c) \impl (a) and completing the proof.
\done

\textbf{Lemma 1.7:}
Let \( R \) be a ring and let \( I \) and \( J \) be ideals of \( R \) such
that \( I + J = R \), then we have that \( I \cap J = IJ \) and
\( R/IJ \cong R/I \times R/J \).

\proof
We have that \( IJ \subseteq I \) since \( I \) absorbs multiplication from
\( J \).
Similarly, \( IJ \subseteq J \), and thus \( IJ \subseteq I \cap J \).
Fix an element \( x \in I \cap J \).
By assumption, \( I + J = R \), and so there exist elements \( a \) and \( b \)
in \( I \) and \( J \), respectively, such that \( a + b = 1 \), hence we
obtain the equality \( x = xa + xb \).
This means that \( x \) is an element of \( IJ \), and thus
\( IJ \subseteq I \cap J \).
A a result, we have \( IJ = I \cap J \).

Next, we consider the ring homomorphism \( \theta : R \to R/I \times R/J \),
which we define as \( \theta(a) = (a + I, a + J) \).
Since \( a + I = I \) and \( a + J = J \) if and only if \( a \in I \cap J \),
we have that \( \ker\theta = I \cap J = IJ \).
This shows that, if \( a + IJ \) and \( b + IJ \) are equivalent elements of
\( R/IJ \), then \( \theta(a) = \theta(b) \).
We can thus define a function \( \phi : R/IJ \to R/I \times R/J \), defined
as \( \phi(a + IJ) = \theta(a) = (a + I, a + J) \).
It is easy to verify that \( \phi \) is a homomorphism.
Additionally, if \( a \in \ker\theta \), then
\( \phi(a + IJ) = \theta(a) = (I,J) \), which shows that
\( \ker\phi = \braceb{0 + IJ} \), and thus \( \phi \) is injective.
Finally, fix \( (y + I, z + J) \in R/I \times R/J \), and again take \( a \)
and \( b \) as elements in \( I \) and \( J \), respectively, such that
\( a + b = 1 \).
Defining the element \( x \in R \) as \( x = az + by \), we can see that
\[
	x \equiv by \equiv (1 - a)y \equiv y - ay \equiv y \pmod{a},
\]
as well as
\[
	x \equiv az \equiv (1 - b)z \equiv z - bz \equiv z \pmod{b}.
\]
This shows that
\[
	\phi(x + IJ) = \theta(x) = (x + I, x + J) = (y + I, z + I),
\]
thus \( \phi \) is surjective, hence an isomorphism, and
\( R/IJ \cong R/I \times R/J \).
\done

\textbf{Lemma 1.8:}
Let \( R \) be a ring and let \( I_1, I_2, \dots, I_n \) be ideals of \( R \)
such that \( I_i + I_j = R \) for all \( 1 \leq i < j \leq n \).
We have that
\( A/I_1I_2 \cdots I_n \cong A/I_1 \times A/I_2 \times \cdots \times A/I_n \).

\proof
The previous lemma proves the case for \( n = 2 \).
We now use induction on \( n \) and assume the case for \( n - 1 \) holds.
Define the ideal \( J = I_2I_3 \cdots I_n \) in \( R \).
Note that for \( 2 \leq k \leq n \), we have \( J \subseteq I_k \) and
\( I_1 + I_k = R \), thus we can find elements \( a_k \in I_1 \) and
\( b_k \in J \) such that \( a_k + b_k = 1 \), and thus
\[
	1 = \prod_{k = 2}^n (a_k + b_k).
\]
Using this we obtain the equality \( 1 = c + a_2a_3 \cdots a_n \), where
\( c \) is the sum of the terms in the product that contain at least one
\( b_k \).
This implies that \( c \in I_1 \), and thus
\( 1 = c + a_2a_3 \cdots a_n \in I_1 + J \).
Because \( I_1 + J \) is an ideal that contains 1, we know that
\( I_1 + J = R \).

By the previous lemma, we have
\[
	A/I_1J \cong A/I_1 \times A/J,
\]
and invoking the induction hypothesis, we can see that
\[
	A/J
	= A/I_2I_3 \cdots I_n
	\cong A/I_2 \times A/I_3 \times \cdots \times A/I_n,
\]
thus we have that
\[
	A/I_1I_2 \cdots I_n
	= A/I_1J
	\cong A/I_1 \times A/J
	\cong A/I_1 \times A/I_2 \times \cdots \times A/I_n,
\]
completing the proof.
\done

\textbf{Proposition 1.9:}
Let \( m \) and \( n \) be relatively prime integers, then we have that
\( \Z/(mn)\Z \cong \Z/m\Z \times \Z/n\Z \).

\proof
By Bezout's lemma, there exist integers \( x \) and \( y \) such that
\[
	xm + yn = 1,
\]
which means that the ideal \( m\Z + n\Z \) contains 1, and thus is equal
to \( \Z \).
Applying \textbf{Lemma 1.7}, we obtain
\[
	\Z/(mn)\Z
	= \Z/(m\Z n\Z)
	\cong \Z/m\Z \times \Z/n\Z.
\]
\done

\textbf{Corollary 1.10:}
If \( m \) and \( n \) are relatively prime, then
\( \phi(mn) = \phi(m)\phi(n) \).

\proof
We previously established that \( \phi(mn) \) is equal to the number of units
in \( \Z/mn\Z \), and thus the number of units in \( \Z/m\Z \times \Z/n\Z \).
Since an element \( (a,b) \in \Z/m\Z \times \Z/n\Z \) is a unit if and only if
\( a \) and \( b \) are units in their respective rings, we have exactly
\( \phi(m)\phi(n) \) choices for units in \( \Z/m\Z \times \Z/n\Z \), thus
proving the equality.
\done

\textbf{Corollary 1.11:}
For fixed \( n \in \N \), let \( p_1^{a_1}p_2^{a_2} \cdots p_m^{a_m} \) be its
prime factorization, then we have that
\[
	\phi(n) = n\prod_{k = 1}^m \parenb{1 - \frac{1}{p_k}}.
\]

\proof
Since powers of distinct primes are always relatively prime, we have that
\( p_i^{a_i}\Z + p_j^{a_j}\Z = \Z \) for all \( 1 \leq i < j \leq m \).
Additionally, \( \Z/n\Z \) has \( \phi(n) \) units.
Furthermore, we have that
\[
	\Z/n\Z
	\cong \Z/p_1^{a_1}\Z p_2^{a_2}\Z \cdots p_m^{a_m}\Z
	\cong \Z/p_1^{a_1}\Z
	\times \Z/p_2^{a_2}\Z
	\times \cdots \times \Z/p_m^{a_m}\Z,
\]
but the number of units in the right hand side is equal to
\( \phi(p_1^{a_1})\phi(p_2^{a_2}) \cdots \phi(p_m^{a_m}) \).
Thus, we have that
\[
	\phi(n)
	= \prod_{k = 1}^m \phi(p_k^{a_k})
	= \prod_{k = 1}^m p_k^{a_k} - p_k^{a_k - 1}
	= \prod_{k = 1}^m p_k^{a_k}\parenb{1 - \frac{1}{p_k}}
	= \prod_{k = 1}^m p_k^{a_k} \prod_{k = 1}^n \parenb{1 - \frac{1}{p_k}}
\]
\[
	= n\prod_{k = 1}^m \parenb{1 - \frac{1}{p_k}}.
\]
\done

\section*{1.4 Preliminaries on Modules}

\textbf{Proposition 1.12:}
Let \( R \) be a ring, \( I \) an indexing set, and \( M \) an \( R \)-module.
Additionally, let \( \braceb{m_i}_{i \in I} \) be a fixed collection of
elements in \( M \), and let \( a = \braceb{a_i}_{i \in I} \) be an element in
\( R^{(I)} \).
We have that the function \( \phi : R^{(I)} \to M \), defined as
\[
	\phi(a) = \sum_{i \in I} a_im_i,
\]
is an \( R \)-module homomorphism.
Furthermore, the following are true:
\begin{enumerate}[(a)]
	\item \( \braceb{m_i}_{i \in I} \) is linearly independent if and only if
	      \( \phi \) is injective

	\item \( \braceb{m_i}_{i \in I} \) generates \( M \) if and only if
	      \( \phi \) is surjective
\end{enumerate}

\proof
Let \( a = \braceb{a_i}_{i \in I} \) and \( b = \braceb{b_i}_{i \in I} \) be
elements in \( R^{(I)} \).
We can see that
\[
	\phi(a + b)
	= \sum_{i \in I} (a_i + b_i)m_i
	= \sum_{i \in I} a_im_i + \sum_{i \in I} b_im_i
	= \phi(a) + \phi(b),
\]
and for \( r \in R \), we have that
\[
	\phi(ra)
	= \sum_{i \in I} ra_im_i
	= r\sum_{i \in I} a_im_i
	= r\phi(a).
\]
Note that we can factor the \( r \) out of the sum since only finitely many
terms are nonzero.
Thus, we have shown that \( \phi \) is an \( R \)-module homomorphism.

Next, assume that the \( m_i \) are linearly independent.
Letting \( a, b \in R^{(I)} \) where \( \phi(a) = \phi(b) \), we see that
\[
	\sum_{i \in I} a_im_i = \sum_{i \in I} b_im_i,
\]
and thus
\[
	\sum_{i \in I} (a_i - b_i)m_i = 0.
\]
The linear independence of the \( m_i \) force \( a_i - b_i = 0 \) for all
\( i \in I \), hence \( a_i = b_i \) and \( a = b \), which shows the
injectivity of \( \phi \).
Conversely, assume \( \phi \) is injective.
It is easy to see that \( \phi(0) = \braceb{0}_{i \in I} \), as
\[
	\phi(0) = \sum_{i \in I} 0m_i = 0.
\]
Additionally, if we have \( a \in R^{(I)} \) such that \( \phi(a) = 0 \), then
by the injectivity of \( \phi \) we have that \( a = 0 \), and thus
\( a_i = 0 \) for all \( i \in I \).

Moving on to (b), we assume that \( \braceb{m_i}_{i \in I} \) generates
\( M \).
This means that, for all \( m \in M \), we have \( m = \sum_{i \in I} a_im_i \)
for

\end{document}
